\documentclass[11pt]{article}

\usepackage{fullpage}
\usepackage{enumitem}

\begin{document}

\title{ARM Final Report}
\author{
  Abhinav Mishra,
  Arthur-Mihai Niculae,
  Szilveszter Szekely,
  Tom Bellingham
}

\maketitle

\section{Introduction}

\section{Emulator}

\section{Assembler}

\subsection{Design}

The assembler has been implemented according to the two pass design provided in
the specification file. However we read the input file only once and iterate
over the stored instructions twice.

The assembler has four major components:
\begin{itemize}[noitemsep,topsep=0pt]
  \item \textbf{Tokenizer}:
    to tokenize the instruction.
  \item \textbf{Parser}:
    to parse the tokenized instruction.
  \item \textbf{Dictionary}:
    to look for the mnemonics and their respective values.
  \item \textbf{Instruction Generator}:
    to generate the final binary encoding.
\end{itemize}

\subsection{Implementation}

Every line in the instruction is fed into the tokenizer which returns a list
of tokens. These tokens are then passed to the parser which recognizes the
mnemonic and parses the whole instruction into a usable format. While the
instruction is being parsed the dictionary is used extensively to classify the
tokens. Once the instruction has been parsed, it is passed to the instruction
generator which connects all the components of the parsed tokens and generates
a final binary encoding of the instruction.

In this implementation of the assembler, majority of the data structures are
being reused from the emulator.

\subsection{Extensions}

The assembler has been extended to implement some more instructions comments.
All the instructions are now of the type:
\textbf{opcode\{cond\} \{operands\}}
\begin{itemize}[noitemsep,topsep=0pt]
  \item \textbf{opcode}: Basic Instruction
  \item \textbf{cond}: Ex: eq, hi, al, cc, etc.
  \item \textbf{operands}: Ex: registers, immediate values.
\end{itemize}
Example: andeq r0, r0, r0, mulhi r1, r2, r3.

It also supports comments along with the instructions which can be used to
explain the code.

\section{Extension}

\subsection{Acknowledgement}

The group takes this opportunity to gratefully acknowledge the people whose
libraries and tutorials have been used to help make this extension.

\begin{itemize}[noitemsep,topsep=0pt]
  \item \textbf{Cambridge University Tutorials} for Baking Pi –
    Operating Systems Development.
  \item \textbf{Valvers.com} for Bare Metal Programming in C.
  \item \textbf{Brian Sidebotham} for Raspberry Pi Baremetal Libraries.
  \item \textbf{PiFox Team} for imager.py utility
  \item \textbf{D. Richard Hipp} for makeheaders utilility
\end{itemize}


\subsection{Idea}

\subsection{Implementation}

\subsection{Result}

\section{Group Reflection}

\section{Personal Reflection}

\subsection{Abhinav Mishra}

\subsection{Arthur-Mihai Niculae}

\subsection{Szilveszter Szekely}

\subsection{Tom Bellingham}

\end{document}
