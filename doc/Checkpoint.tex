\documentclass[11pt]{article}

\usepackage{fullpage}

\begin{document}

\title{ARM Checkpoint}
\author{
  Szilveszter Szekely,
  Abhinav Mishra,
  Arthur-Mihai Niculae,
  Tom Bellingham
}

\maketitle

\section{Group Organisation}

Our idea was to split up the parts of the task among the different members of
the group, first focusing on the emulator and the assembler and then the
remaining GPIO LED flasher and extension parts.

The emulator was assigned to Szilveszter Szekely and Arthur-Mihai Niculae.
The assamble was assigned to Abhinav Mishra and Tom Bellingham.

At this point we have yet to decide on the specifics of the work allocation of
GPIO LED program and the extension.

\section{Group Dynamics}

Our group is host to a wide range of abilities and expertise, which has provided
a unique opportunity to query and learn from those who have considerable skill
in particular areas.

For some of us, the C lectures have been our first exposure to the language,
thus studying the language independently in order to contribute to the group
more effectively later on has been a hindrance to working as a team in the early
stages.

Partitioning the group into halves, with each pair tackling one of the first two
tasks, has so far been an effective strategy. However, given the discrepancy in
knowledge and ability between group members, some members were able to
contribute disproportionally to the end result.

In these situations we have peer-reviewed each other's code to discuss and
ensure that we understand the design and implementation decisions that were
made.

In future we need to ensure that we communicate more systematically and
regularly about progress with specific tasks, and strategize not only about
immediate objectives but also upcoming tasks and the long-term life of the
project.

\section{Implementation Strategies}

We wanted a consistent style for our code and one that also filters out most of
the possible errors that we can make. Therefore we decided to include the Linux
kernel checkpatch.pl tool to check our for style consistency. We also turned on
more warnings in our Makefile to spot more problems. We made any style error and
warning along with compiler errors into errors that stop the build, so we can
spot and correct them easily.

The emulator was split into four main parts.

There are the data structures for the storing the state of the emulator and the
decoded instructions. The state includes the current values of registers, the
current values in memory, whether there is a fetched instructions and the
fetched instruction, whether there is a decoded instruction and the decoded
instruction itself.

The decoded instructions are stored in a single structure that has a flag
indicating the kind of the instruction and the specific parts to their their
respective instructions. The are stored in an union in order to save on memory.

The remaining four parts are split based on the parts of the execution pipeline,
therefore fetch, decode, execute.

Fetch is relatively simple, as it only needs to fetch from memory and copy four
bytes into a 32 bit instruction value and set the instruction fetched flag.

Decoding the instructions consist of two steps. First, the kind of the
instruction is identified using bit patterns that we have found to uniquely
specify them. Then the specifics of each instruction kind are parsed into the
instruction structure.

Execution the instruction happens by having conditions on which parts of the
structure are filled in. First the condition is checked then if it needs to be
the stored instruction is executed, doing the necessary processing on them like
doing the value rotations, extensions and the like.

Finally the PC register is incremented.

If any of these operations fail or a halt instruction is encountered then
execution is halted and the state of the emulator dumped.

To simplify code paths, we use arrays of function pointers and index into them
with relevant parts of the instruction, for example the opcode of the data
processing instruction.

The data structures of the emulator are to be reused in the assembler.

TODO Assembler structure, maybe.

\section{Implementation Challenges}

TODO A discussion on implementation tasks that you think you will find difficult
/ challenging later on, and how you are working to mitigate these.

TODO What to write here?...

\end{document}
