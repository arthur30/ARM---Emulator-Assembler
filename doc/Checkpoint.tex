\documentclass[11pt]{article}

\usepackage{fullpage}
\usepackage{enumitem}

\begin{document}

\title{ARM Checkpoint}
\author{
  Szilveszter Szekely,
  Abhinav Mishra,
  Arthur-Mihai Niculae,
  Tom Bellingham
}

\maketitle

\section{Group Organisation}

Our idea was to split up the parts of the task among the different members of
the group, first focusing on the emulator and the assembler and then the
remaining GPIO LED flasher and extension parts.

The emulator was assigned to Szilveszter Szekely and Arthur-Mihai Niculae.
The assembler was assigned to Abhinav Mishra and Tom Bellingham.

At this point we have yet to decide on the specifics of the work allocation of
GPIO LED program and the extension.

\section{Group Dynamics}

Our group is host to a wide range of abilities and expertise, which has provided
a unique opportunity to query and learn from those who have considerable skill
in particular areas.

For some of us, the C lectures have been our first exposure to the language,
thus studying the language independently in order to contribute to the group
more effectively later on has been a hindrance to working as a team in the early
stages.

Partitioning the group into halves, with each pair tackling one of the first two
tasks, has so far been an effective strategy. However, given the discrepancy in
knowledge and ability between group members, some members were able to
contribute disproportionally to the end result.

In these situations we have peer-reviewed each other's code to discuss and
ensure that we understand the design and implementation decisions that were
made.

In future we need to ensure that we communicate more systematically and
regularly about progress with specific tasks, and strategize not only about
immediate objectives but also upcoming tasks and the long-term life of the
project.

\section{Implementation Strategies}

We wanted a consistent style for our code and one that also filters out most of
the possible errors that we can make. Therefore we decided to include the Linux
kernel checkpatch.pl tool to check our for style consistency. We also turned on
more warnings in our Makefile to spot more problems. We made any style error and
warning along with compiler errors into errors that stop the build, so we can
spot and correct them easily. \\ \\

The \textbf{emulator} was split into four main parts.

There are data structures for storing the state of the emulator and the
decoded instructions. The state includes the current values of registers, the
current values in memory, whether there is a fetched instructions and the
fetched instruction, whether there is a decoded instruction and the decoded
instruction itself.

The remaining four parts are split based on the parts of the execution pipeline,
therefore fetch, decode, execute.

\begin{itemize}[noitemsep,topsep=0pt]
    \item \textbf{Fetch}: relatively simple, fetches 4 bytes from memory, packs them
    \item \textbf{Decode}: tests for specific bit patterns to identify instruction type
      then unpacks it
    \item \textbf{Execute}: check the flags on the decoded instruction and calls their
      respective functions by indexing into an array
    \item \textbf{PC}: incremented
\end{itemize}

If any of these operations fail or a halt instruction is encountered then
execution is halted and the state of the emulator dumped.

To simplify code paths, we use arrays of function pointers and index into them
with relevant parts of the instruction, for example the opcode of the data
processing instruction. To save space the instructions are stored in unions.

The data structures of the emulator are to be reused in the assembler. \\

For the \textbf{assembler} we are using the two pass strategy mentioned in the
spec file. In the first pass we read the labels and in the second pass we write
the binary encoded instruction.

Talking about how the assembler generates this binary encoding. The assembler
has been divided into 4 main parts:
\begin{itemize}[noitemsep,topsep=0pt]
	\item \textbf{Tokenizer}:
		to tokenize the instruction.
	\item \textbf{Parser}:
		to parse the tokenized instruction.
	\item \textbf{Dictionary}:
		to look for the mnemonics and their respective values.
	\item \textbf{Instruction Generator}:
		to generate the final binary encoding.
\end{itemize}

We also reuse some of the data structures used in the emulator in order to
avoid duplicates and redundant code.(Ex:
\textbf{instr\_data\_proc}, \textbf{instr\_mult}, \textbf{instr\_transfer} \& \textbf{instr\_branch}
)

We would also like to mention that there is some work to be done on the
assembler to pass all the test cases. Also we are thinking of reducing the
number of components underlying the assembler and making the tokenizer more
efficient and complete.

\section{Implementation Challenges}

As was mentioned in the previous part we used very strong rules about checking
code style. Therefore some times we had to search for new methods to do a part
that we were warned about being unsafe.

Coming from 6 months of Java some of us had a tough time with memory management
in C. Receiving SIGSEGV error every now and then was not very pleasant to the
eyes and the mind. So we had to be very careful with how and when to allocate
memory and when to free it.

Also we found shift registers hard to implement in the assembler because of an
inefficient tokenizer. So we had to rewrite it in order to implment the
shifting of registers.

\end{document}
